\documentclass[12pt,letterpaper, onecolumn]{exam}
\usepackage{amsmath}
\usepackage{amssymb}
\usepackage{gensymb}
\usepackage[utf8]{inputenc}
\usepackage[lmargin=71pt, tmargin=1.2in]{geometry}
\usepackage{circuitikz}
\thispagestyle{empty}

\begin{document}

% ===================== HEADER ======================
\begingroup  
    \centering
    \LARGE ECE 210 Review\\
    \LARGE Midterm Two Fall 2024\\[0.5em]
    \large \today\\[0.5em]
    \large Created by HKN\par
\endgroup
\rule{\textwidth}{0.4pt}

\pointsdroppedatright
\printanswers
\renewcommand{\solutiontitle}{\noindent\textbf{}\enspace}

\newcommand\myNIA[4]{%
    \draw #2 coordinate(#1-in) to[short] ++(1,0)
    node[op amp, noinv input up, anchor=+](#1-OA){\texttt{#1}}
    (#1-OA.-) -- ++(0,-1) coordinate(#1-FB)
    to[R=#3] ++(0,-2) node[ground]{}
    (#1-FB) to[R=#4, *-] (#1-FB -| #1-OA.out) -- (#1-OA.out)
    to [short, *-] ++(1,0) coordinate(#1-out)
    ;
}

% ===================== SECTION 1 ======================
\section{[25 points] The three parts of this problem can be solved independently.}

\begin{questions}

\question Determine the phasor of the sinusoidal signal 
\[
f(t) = 10\sin\left(\frac{\pi}{4}t - \frac{3\pi}{4}\right).
\]
Express it both in polar and rectangular.  
\textbf{(5 points)}  
\begin{solution}
    \begin{parts}
        \part Polar:
        \part Rectangular:
    \end{parts}
\end{solution}

\question Determine the phasor of the sinusoidal signal  
\[
f(t) = 4\cos\left(5t + \frac{\pi}{4}\right) + 12\cos\left(t + \frac{7\pi}{4}\right).
\]
Express it in either polar or rectangular form.  
\textbf{(5 points)}  
\begin{solution}
\end{solution}

\question Given the LTI system with frequency response  
\[
H(\omega) = \frac{2}{12 + j\omega}
\]
and the input  
\[
f(t) = 7 + \sin\left(5t + \frac{\pi}{2}\right) + \cos\left(12t + \frac{\pi}{4}\right),
\]
determine the output \(y(t)\).  
\textbf{(25 points)}  
\begin{solution}
\end{solution}

\end{questions}

% ===================== SECTION 2 ======================
\section{[15 points] Consider the circuit shown below.}

\begin{center}
\begin{circuitikz}[scale=0.9, transform shape]
    \draw (0,0) node[ground]{}
          to[sV, v=$f(t)$] (0,2)
          to[R=$2\,\Omega$] (3,2)
          to[C=$1\,\mathrm{F}$, -*] (3,0) node[ground]{};
    \draw (3,2) node[above] {$v(t)$};
\end{circuitikz}
\end{center}

For \(t > 0\), the source voltage is
\[
f(t) = 4 e^{-t} u(t)\ \text{V},
\]
and the capacitor has initial value \(v(0^-) = 1\ \text{V}\).

\begin{questions}

\question Derive the differential equation (ODE) governing the capacitor voltage \(v(t)\) for \(t > 0\).  
\textbf{(7 points)}  
\begin{solution}
\end{solution}

\question Solve for \(v(t)\) for \(t > 0\), using the initial condition.  
\textbf{(8 points)}  
\begin{solution}
\end{solution}

\end{questions}

\end{document}
