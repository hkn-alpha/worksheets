\documentclass[12pt,letterpaper, onecolumn]{exam}
\usepackage{amsmath}
\usepackage{amssymb}
\usepackage{gensymb}
\usepackage[utf8]{inputenc}
\usepackage[lmargin=71pt, tmargin=1.2in]{geometry}  %For centering solution box
\usepackage{circuitikz}
% \lhead{Leaft Header\\}
% \rhead{Right Header\\}
% \chead{\hline} % Un-comment to draw line below header
\thispagestyle{empty}   %For removing header/footer from page 1

\begin{document}

\begingroup  
    \centering
    \LARGE ECE 210 Review\\
    \LARGE Midterm Two Fall 2024\\[0.5em]
    \large \today\\[0.5em]
    \large Created by HKN\par
\endgroup
\rule{\textwidth}{0.4pt}
\pointsdroppedatright   %Self-explanatory
\printanswers
\renewcommand{\solutiontitle}{\noindent\textbf{}\enspace}   %Replace "Ans:" with starting keyword in solution box

\newcommand\myNIA[4]{%1: name of this amplifier, %2 start coordinate, %3 R1, %4 R2
    \draw #2 coordinate(#1-in) to[short] ++(1,0)
    node[op amp, noinv input up, anchor=+](#1-OA){\texttt{#1}}
    (#1-OA.-) -- ++(0,-1) coordinate(#1-FB)
    to[R=#3] ++(0,-2) node[ground]{}
    (#1-FB) to[R=#4, *-] (#1-FB -| #1-OA.out) -- (#1-OA.out)
    to [short, *-] ++(1,0) coordinate(#1-out)
    ;
}


\section{[25 points] The three parts of this problem can be solved independantly.}

\begin{questions}

    \question[5 points] Determine the phasor of the sinusoidal signal \begin{math}f(t) = 10\sin{(\frac{\pi}{4}t - \frac{3\pi}{4})}\end{math}? Express it both in polar and rectangular. \droppoints


    
    
    \begin{solution}
        \begin{parts}
            \part Polar: 
            \part Rectangular: 
        \end{parts}
    \end{solution}
    
    \question[5 points] Determine the phasor of the sinusoidal signal \begin{math}f(t) = 4\cos{(5t + \frac{\pi}{4})} + 12\cos{(t + \frac{7\pi}{4})}\end{math}? Express it either polar or rectangular form. \droppoints
    
    \begin{solution}
    \end{solution}


    \question[25 points] Given an LTI system with a frequency response \begin{math}
        H(\omega) = \frac{2}{12 + j\omega}
    \end{math} and an input of \begin{math}
        f(t) = 7 + \sin{(5t + \frac{\pi}{2})} + \cos{(12t + \frac{\pi}{4})}
    \end{math}, determine the output \begin{math}y(t)\end{math}. \droppoints
    
    \begin{solution}
    \end{solution}

    

    % \pagebreak %Not necessary
\end{questions}

\section{[15 points] Provided the following Op-Amp circuit, answer the following two questions}.

    \begin{circuitikz}[scale=0.7, transform shape
    ]
    \myNIA{OA1}{(0,0)}{$R_1$}{$R_2$}
    \myNIA{OA2}{(OA1-out)}{$R_3$}{$R_4$}
    \node [ocirc] at (OA1-in) {};
    % \node [above] at (OA1-in) {$v_i$};
    \node [ocirc] at (OA2-out){};
    
    \node [ocirc] at (-4, 0){};
    \node [ground] at (-4, -4){};
    \node [ocirc] at (-4, -2){};
    \node [ocirc] at (-2, 0){};
    \draw (OA1-out) -| (OA2-in);
    \draw (-4, -4) -- (-4, -2) to[sV, v=$v_s$] (-4, 0) -- (-2,0) to[R=$R_3$, *-] (OA1-in);
    
    
    \end{circuitikz}
    
\begin{questions}

    \question[10 points] Determine the voltage \begin{math}v(t)\end{math} of this circuit.\droppoints    
    
\end{questions}


\section{Solutions}
\begin{enumerate}
    \item{Determine the phasor of the sinusoidal signal \begin{math}f(t) = 10\sin{(\frac{\pi}{4}t - \frac{3\pi}{4})}\end{math}? Express it both in polar and rectangular.\\}

        To solve this question, let's first transform the sine function to a cosine function. To do so, we can subtract \begin{math}\frac{\pi}{2}\end{math}, or 90\degree, from the phase:
    
        \begin{center}\begin{math}
            f(t) = 10\sin{(\frac{\pi}{4}t - \frac{3\pi}{4})} =
            10\cos{(\frac{\pi}{4}t - \frac{3\pi}{4} - \frac{\pi}{2})}
            = 10\cos{(\frac{\pi}{4}t - \frac{5\pi}{4})}
        \end{math}\end{center}
    
        From this form, we get that the polar form of this function is \begin{math}10e^{-\frac{5\pi}{4}j} \end{math}. Expressing this in rectangular form, we have \begin{math}10 * \frac{-\sqrt{2}}{2} + 10j * \frac{\sqrt{2}}{2}\end{math}, or \begin{math}-5\sqrt{2} + 5\sqrt{2}j\end{math}.

    
    
    \item {Determine the phasor of the sinusoidal signal \begin{math}f(t) = 4\cos{(5t + \frac{\pi}{4})} + 12\cos{(t + \frac{7\pi}{4})}\end{math}? Express it either polar or rectangular form.\\}

        The first step to solving this question is to directly transform each part of the equation into it's rectangular form! Let's break down it down:

        \begin{center}\begin{math}
            4\cos{(5t + \frac{\pi}{4})} = 4 * \frac{\sqrt{2}}{2} + 4j* \frac{\sqrt{2}}{2} = 2\sqrt{2} + 2\sqrt{2}j            
        \end{math}\end{center}
        \begin{center}\begin{math}
           12\cos{(t + \frac{7\pi}{4})} = 12 * \frac{\sqrt{2}}{2} - 12j* \frac{\sqrt{2}}{2} = 6\sqrt{2} - 6\sqrt{2}j
        \end{math}\end{center}

        With this, we get the rectangular form of this equation to be \begin{math}
            2\sqrt{2} + 2\sqrt{2}j + 6\sqrt{2} - 6\sqrt{2}j
        \end{math}, or \begin{math}
            F = 8\sqrt{2} - 4\sqrt{2}j
        \end{math}. To get the polar coordinates, we need to find the magnitude of our function and it's phase in radians:
        \begin{center}\begin{math}
            |F| = \sqrt{(8\sqrt{2})^2+(-4\sqrt{2})^2} = \sqrt{128 + 32} = 4\sqrt{10}
        \end{math}\end{center}
        \begin{center}\begin{math}
            \angle F = \arctan{\frac{-4\sqrt{2}}{8\sqrt{2}}} = \arctan{\frac{-1}{2}}
        \end{math}\end{center}

        Combining these together, you get the polar coordinate form of this expression to be \begin{math}F = 4\sqrt{10}*e^{\arctan{(\frac{-1}{2})} }\end{math}.


    \item {Given an LTI system with a frequency response \begin{math}
        H(\omega) = \frac{2}{12 + j\omega}
    \end{math} and an input of \begin{math}
        f(t) = 7 + \sin{(5t + \frac{\pi}{2})} + \cos{(12t + \frac{\pi}{4})}
    \end{math}, determine the output \begin{math}y(t)\end{math}.\\}

        To solve this problem, we will break this problem into three parts, corresponding to each of the aspects of \begin{math}Y(\omega)\end{math}.

        The first part of \begin{math}Y(\omega)\end{math} is the constant. That can be expanded out to \begin{math}7\cos{0t}\end{math}. Plugging this new omega value into our \begin{math}H(\omega)\end{math}, we get that \begin{math}H(0) = \frac{2}{12}\end{math}, or \begin{math}\frac{1}{6}\end{math}. Considering there is no additional phase for a constant, this completes the first part.

        The second part of \begin{math}Y(\omega)\end{math} is \begin{math}\sin{(5t + \frac{\pi}{2})}\end{math}. To make it easier for us, we will transform the sine wave to cosine wave. 

        \begin{center}\begin{math}
            \sin{(5t + \frac{\pi}{2})} = \cos{(5t + \frac{\pi}{2} - \frac{\pi}{2})} = \cos{(5t)}
        \end{math}\end{center}

        Plugging the omega coefficient into our \begin{math}
            H(w)
        \end{math}, we get \begin{math}
            H(5)= \frac{2}{12 + 5j}
        \end{math}. To determine the output of this frequency, we need to transform this into either it's rectangular or polar form. To get it's polar form, we will calculate the magnitude and phase of \begin{math}
            H(w)
        \end{math}.

        \begin{center}\begin{math}
            |H(5)| = \sqrt{\frac{2^2}{12^2 + 5^2}} = \sqrt{\frac{2}{13}}
        \end{math}\end{center}
        \begin{center}\begin{math}
            \angle H(5) = \frac{e^0}{e^{\arctan{(\frac{5}{12})}}} = e^{-\arctan          {(\frac{5}{12})}}
        \end{math}\end{center}

        With this, we get that \begin{math}\sin{(5t + \frac{\pi}{2})}\end{math} transforms to \begin{math}\sqrt{\frac{2}{13}}e^{-\arctan{(\frac{5}{12})}}\end{math}.

        The final part of \begin{math}Y(\omega)\end{math} is \begin{math}\cos{(12t + \frac{\pi}{4})}\end{math}. We'll first evaluate the omega coefficient into \begin{math}H(\omega)\end{math}.

        \begin{center}\begin{math}
            H(12) = \frac{2}{12 + 12j} = \frac{1}{6 + 6j} = \frac{1}{6\sqrt{2}e^{\frac{\pi}{4}}} = \frac{e^{\frac{-\pi}{4}}}{6\sqrt{2}}
        \end{math}\end{center}

        Adding in the original phase of \begin{math}\frac{\pi}{4}\end{math}, we get \begin{math}\cos{(12t + \frac{\pi}{4})}\end{math} transforms into \begin{math}\frac{e^{\frac{-\pi}{4}}}{6\sqrt{2}}*e^{\frac{\pi}{4}} = \frac{1}{6\sqrt{2}}\end{math}. Combining all the various aspects together, we get the full \begin{math}Y(w)\end{math} to be:

            \begin{center}\begin{math}
                Y(\omega) = \frac{1}{6} + \sqrt{\frac{2}{13}}e^{-\arctan{(\frac{5}{12})}} + \frac{1}{6\sqrt{2}} = \frac{2 + \sqrt{2}}{12} + \sqrt{\frac{2}{13}}e^{-\arctan{(\frac{5}{12})}}
            \end{math}\end{center}

        Transforming this \begin{math}Y(\omega)\end{math} back into the time domain, we get our \begin{math}y(t)\end{math} to be:

            \begin{center}\begin{math}
                y(t) = \frac{2 + \sqrt{2}}{12} + \sqrt{\frac{2}{13}}\cos{(5t - {\arctan{(\frac{5}{12})}})}
            \end{math}\end{center}

\end{enumerate}
\pagebreak %Not necessary

\end{document}