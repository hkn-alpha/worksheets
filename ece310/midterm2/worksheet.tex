\documentclass{article}
\usepackage{graphicx}
\usepackage{amsmath}

\title{HKN ECE 310 Review Worksheet 2}
\author{Author: Members of HKN }
\date{}
\usepackage[makeroom]{cancel}
\usepackage[letterpaper, portrait, margin=1in]{geometry}
\pagenumbering{arabic}

\begin{document}

\maketitle

\section{The very basics}
\begin{enumerate}
    \item What is the relation between the Z-transform and the Discrete-Time Fourier Transform? When is this relation not valid?
    $$ X_{d}(\omega) = X(z)\vert_{z=\exp(j\omega)} $$

\end{enumerate}

\section{Sampling and DTFTs}
(Let the output of a radio be) Consider a signal given by
$$
x(t) = 2\cos(10\pi t) + \sin(30\pi t)
$$
\begin{enumerate}
    \item What is the nyquist sampling rate of this signal?
    
    Since the maximum frequency of any of the components of this signal is 15Hz, the minimum sampling rate must be $2\cdot 15$Hz $= 30$Hz.

    \item Let's say the signal is sampled at twice the nyquist rate. What does the discrete-time signal look like for three samples starting at $n=0$? What is the $n$'th sample?

    We are sampling at $60$Hz, so the sampled signal is
    $$x[n] = 2\cos\left( \frac{1}{6}\pi n \right) + \sin\left( \frac{1}{2}\pi n \right)$$
   
    \item Find the Discrete-Time Fourier Transform of this signal. Plot both the real and imaginary components of the DTFT over the range $(-\pi, \pi)$.

    The DTFT is
    $$X_{d}(\omega) = 2\pi \left[ \delta\left(\omega - \frac{\pi}{6}\right) + \delta\left(\omega + \frac{\pi}{6}\right) \right] - j \pi \left[ \delta\left(\omega - \frac{\pi}{2}\right) - \delta\left(\omega + \frac{\pi}{2}\right) \right]$$

    \item What is the power contained in this signal? Make sure to include units!

    Using Parseval's Relation, the power is 5$\pi$ watts.

    \item If we want to build a low-pass filter to filter out the fastest component of this signal, what is the smallest value of $\omega$ at which the filter can start attenuating? 

    $$\omega_\text{cutoff} = \frac{\pi}{6}$$

    \item Lets say we have a perfect filter to do said filtering. We then amplify the signal such that the magnitude of each component is doubled. How does the power of the signal change?

    Using Parseval's relation again, the power contained is $16\pi$ watts.
    
\end{enumerate}

\end{document}
